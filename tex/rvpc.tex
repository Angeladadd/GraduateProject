% !TeX root = ../thesis.tex

\chapter{随机传值进程模型}

由于Uniform Approach是模型无关的概率扩展方法,
我们可以使用该方法概率化扩展其他的进程模型。
本章会对传值进程模型进行概率化扩展。
传值进程模型是经典的进程模型,
在Milner的Communication and Concurrency一书中,
也是使用传值的通信并发模型——消息的发送者、接受者和通信媒介——引入并发和通信的概念。
我们选择传值进程模型的一种定义——The Value-Passing Calculus进行扩展。

\section{传值进程模型}
传值进程模型(Value-Passing Calculus)是一种
可以将某个域中的值作为通信的内容,
并且可以在特定逻辑条件执行某个动作的并发进程模型。
一个经典的例子是Milner书中的一单元缓冲区,它可以接受、储存、发送消息:
\begin{align*}
   C&\stackrel{def}{=}in(x).C'(x)\\
   C'(x)&\stackrel{def}{=}\overline{out}(x).C
\end{align*}
更“普遍”的传值进程模型也可以对输入进行逻辑判定,
根据特定的“条件”执行特定的“动作”,
通过输入控制进程的执行。我们以进程$A(x)$为例:
$$A(x)=\textrm{if }\varphi(x)\textrm{ then }\overline{a}(f(t))\textrm{ else }B(x)$$
$A(x)$在满足条件$\varphi(x)$时会通过通道$a$输出函数$f(t)$的值,
其中$t$可能是$x$的函数,也可能与$x$无关;若不满足$\varphi(x)$,则执行程序$B(x)$。

传值进程模型赋予了进程传递数据的能力,
我们可以通过进程之间传递的数据来控制进程的执行和进程间的交互,
这种控制能力显著的增强了并发进程模型的表达能力,拓宽了应用场景。
在网络通信中,各种通信协议通过在主机间传值进而控制主机的行为,
通信协议的过程就可以被抽象为传值进程模型,
支持并发的编程语言也可以被解释为传值进程模型,如Erlang。
很多通过传值,计算解决的现实问题也都可以抽象为传值进程模型。


\subsection{The Value-Passing Calculus}
对传值进程模型的研究很多都会依赖一个“神域”,
这个“神域”提供条件的判定,函数的计算和变量的取值范围。
如在执行$A(x)$时,我们会将$\varphi(x)$传给“神域”,
“神域”判定是否满足条件,并将判定结果返回给我们,
同样的,我们也会将$f(t)$传给“神域”,“神域”帮我们计算这个函数,并将结果返回给我们。
“神域”通常不包括在传值进程模型中,且通常是未定义的,或只在特定场景下定义。
由于“神域”的存在,对于这些传值进程模型表达能力的衡量变得十分困难,
Fu在The Value-Passing Calculus中取缔了“神域”的概念,
提出了一个封闭的传值进程模型。

The Value-Passing Calculus中的传值进程模型$\mathbb{VPC}_{\mathsf{Th}}$
的语法可以表示为:
$$T:=\sum_{i\in I}\varphi_i a(x).T_i|\sum_{i\in I}\varphi_i\overline{a}(t_i).T_i|T|T'|(c)T|\varphi T|!a(x).T|!\overline{a}(t).T$$
其中$T$是一个$\mathbb{VPC}_{\mathsf{Th}}$项,$\mathsf{Th}$是可判定的逻辑,
$\varphi_i$是一个布尔表达式,可以通过$\mathsf{Th}$证明或证伪,
若$\mathsf{Th}$可以给出$\varphi_i$的一个证明,记为$\mathsf{Th}\vdash \varphi_i$。
$a(x)$是一个输入前缀,$\overline{a}(t_i)$是一个输出前缀,
我们可以用$\sum_{i\in I}\varphi_i\lambda_i.T_i$来表示$\sum_{i\in I}\varphi_i a(x).T_i$或$\sum_{i\in I}\varphi_i \overline{a}(t).T_i$。
$!a(x).T$和$!\overline{a}(t).T$对应递归操作,$\mu X.E$可以表示为
$(c)(E\{c(z).0/X\}|!\overline{c}(r).E\{c(z).0/X\})$。
其余的表达与第一章CCS语法定义的解释相同。

The Value-Passing Calculus提出了两种转移语义的描述:具体语义(Concrete Semantics)和符号语义(Symbolic Semantics)。
后文我们只用到了符号语义,因此我们忽略具体语义的相关内容。

$\mathbb{VPC}_{\mathsf{Th}}$的符号语义如图~\ref{fig_vpc}所示。

\begin{figure}[!htbp]
   \small
   \centering
   \includegraphics[width=13cm]{../figure/vpc.png}
    \caption[]{$\mathbb{VPC}_{\mathsf{Th}}$的符号语义规则}
    \label{fig_vpc}
\end{figure}
其中,$T=\varphi \lambda.T'$在$\varphi$条件下执行$\lambda$动作到达$T'$状态,
在符号语义下可以写作$T\stackrel{\lambda}{\rightarrow}_{\varphi}T'$。
符号语义中的动作集合$Act=\{a(x),\overline{a}(t)| a\in Chan, x\in \mathsf{V}_{\Sigma}, t\in \mathsf{T}_{\Sigma}\}\cup \{\tau\}$,
$Act$中的元素可以用$\lambda$表示,其中$\mathsf{V}_{\Sigma},\mathsf{T}_{\Sigma}$为可判定逻辑$\mathsf{Th}$中变量的集合和项的集合[引用]。

The Value-Passing Calculus使用可判定的一阶理论判定条件,
提出了一个图灵完备的数值系统(Numeric System)作为底层模型,
用来实现可计算的函数。通过这两种方式,将Value-Passing Calculus从“神域”中解放出来。

\section{随机传值进程模型(Random VPC)}
我们将Uniform Approach中的CCS部分替换为VPC,可以得到Random VPC的语法。
$$T:=\bigoplus_{i\in I}p_i \tau.T_i\mid \sum_{i\in I} \varphi_i\lambda_i.T_i\mid T \mid T'\mid (c)T\mid \varphi T\mid !a(x).T \mid !\bar{a}(t).T$$

$\mathcal{T}_{RVPC}$为所有Random VPC项目的集合。

\subsection{随机传值进程模型转移语义}
Uniform Approach修改了CCS的prefix操作符,
在CCS的基础上增加了随机性行为,$\oplus p\tau.T$体现了通道的随机性。

对VPC的$\bigoplus_{i\in I}p_i \tau.T_i$算子做对应的概率转移,
我们可以得到Random VPC的符号转移语义:

\begin{figure}[!htbp]
	\small
	\centering
	\includegraphics[width=13cm]{../figure/symbolic_sematic.png}
    \caption{\textbf{Symbolic Semantics}}
    \label{fig_sematic}
\end{figure}

在VPC的Symbolic Semantics中,
对于VPC的典型prefix运算$\sum \varphi \lambda. T\stackrel{\lambda}{\rightarrow}_{\varphi} T$,
我们可以通过Uniform Approach的方法将它扩展为一个概率的操作
$\bigoplus p\tau.\varphi \lambda. T\stackrel{p\tau}{\rightarrow}_{\top}\stackrel{\lambda}{\rightarrow}_{\varphi} T$。
它的语义就会变成:在概率$p$下,我们会经过一个内部$\tau$操作到达一个Random VPC状态:$\varphi\lambda.T$,
若$\mathsf{Th}\vdash \varphi$(即一阶理论$\mathsf{Th}$下,$\varphi$为真),
则我们可以经过$\lambda$操作到达Random VPC状态$T$, 其中$\lambda \in \{a(x),\bar{a}(x)\mid a\in \mathcal{N}, x\in \mathsf{V}_\Sigma, t\in \mathsf{T}_\Sigma\}\cup \{\tau\}$。

\subsection{VPC延伸性质与定义}

在CCS,Uniform Approach,VPC中,
对于if then else语法的描述仍然保留着if then else的原始格式,
我们可以根据VPC的定义给出符合VPC转移规则的if then else语法。
以下推论中$S,T\in \mathcal{T}_{\mathbb{VPC}}$。
\begin{corollary} 
   $\varphi 0 = 0$
\end{corollary}
\begin{proof}
   $\varphi 0 = \varphi 0 + 0 = \varphi 0 + \top 0 = \varphi 0 + (\varphi \vee \urcorner \varphi)0 = \varphi 0 + \varphi 0 + \urcorner \varphi 0 = \varphi 0 + \urcorner \varphi 0 = (\varphi \vee \urcorner \varphi)0 = \top 0 = 0$
\end{proof}
\begin{corollary}
   $\varphi T = (\varphi T\mid \urcorner \varphi 0)$
\end{corollary}
\begin{proof}
   $\varphi T = (\varphi T\mid 0) = (\varphi T\mid \urcorner \varphi 0)$
\end{proof}
\begin{corollary}
   $S=\varphi T$,那么当$\mathsf{Th}\vdash \urcorner \varphi$时,$S=0$。
\end{corollary}
进而我们可以得到if then else语法与VPC规则的对照表:
\begin{table}[!hpt]
   \caption[If Then Else语法对照表]{If Then Else语法对照表\footnotemark}
   \label{tab:ifthenelse}
   \centering
   \begin{tabular}{@{}cc@{}} \toprule
   %   \multicolumn{2}{c}{Item} \\ \cmidrule(r){1-2}
     语法 & $\mathbb{VPC}$规则对照 \\ \midrule
     $S=$ if $\varphi$ then $T$ else $T'$& $S=(\varphi T|\urcorner \varphi T')$\\
     $S=$ if $\varphi$ then $T$ & $S=(\varphi T|\urcorner\varphi 0)$\\
     $S = $if $\bot$ then $T$ & $S=0$\\ \bottomrule
   \end{tabular}
 \end{table}

对类似$S\stackrel{p_1\tau}{\rightarrow}_{\varphi_1}\stackrel{p_2\tau}{\rightarrow}_{\varphi_2}\cdots\stackrel{p_n\tau}{\rightarrow}_{\varphi_n} T$
考虑将他们写成这种形式: $S\stackrel{p}{\Rightarrow}_{\varphi} T$。

\begin{definition}\label{def:continous_vpc}
   根据VPC[此处应有引用],若$S,T\in \mathcal{T}_{\mathbb{VPC}}$,$S\stackrel{\tau}{\rightarrow}_{\varphi_1}\stackrel{\tau}{\rightarrow}_{\varphi_2}\dots \stackrel{\tau}{\rightarrow}_{\varphi_n} T$,则可以记成$S\stackrel{}{\Rightarrow}_{\varphi_1\varphi_2\dots\varphi_n} T$。
\end{definition} 

而Uniform Approach中没有对连续的概率转移作类似的缩写定义,
我们根据定义2.1作出相应的定义。
\begin{definition}
   若$S,T\in \mathcal{P}_{RCCS}$,$S\stackrel{p_1\tau}{\rightarrow}\stackrel{p_2\tau}{\rightarrow}\dots\stackrel{p_n\tau}{\rightarrow}T$,则记为$S\stackrel{p_1p_2\dots p_n}{\Longrightarrow}T$。
\end{definition}

\begin{definition}
   若$S,T\in \mathcal{T}_{RVPC}$,
$S\stackrel{p_1\tau}{\rightarrow}_{\varphi_1}\stackrel{p_2\tau}{\rightarrow}_{\varphi_2}\dots \stackrel{p_n\tau}{\rightarrow}_{\varphi_n} T$,则可以记成$S\stackrel{p_1p_2\dots p_n}{\Longrightarrow}_{\varphi_1\varphi_2\dots\varphi_n} T$。
若$\varphi = \varphi_1\varphi_2\dots \varphi_n, p=p_1p_2\dots p_n$,
则可以记为$S\stackrel{p}{\Rightarrow}_{\varphi} T$。
\end{definition} 

\begin{definition}
   条件选择下的符号语义:
   若$S=\varphi (p\tau.T+(1-p)\tau.T')$,
   可以记$S\stackrel{p\tau}{\rightarrow}_{\varphi} T$。
\end{definition} 
\subsection{条件等价集}
 \begin{definition}
   $A,A'\in \mathcal{T}_{RVPC}$,若$\varphi A \mathcal{E} \varphi A'$,则$A'\in [A]_{\varphi \mathcal{E}}$。
   $[A]_{\varphi \mathcal{E}}$称为等价关系$\mathcal{E}$在条件$\varphi$下包含$A$的等价集。  
 \end{definition} 
 定义5的产生是由于:在$\mathbb{VPC}$的符号互模拟(Symbolic Bisimulation)中,
 我们考虑用必要条件下的动作去模拟充分条件下的动作[此处应有引用]。
 举个简单的例子:
 对于$A(x)=(x\geq 5)\tau.B(s(x))$
 和$A'(x)=(x\geq 3)\tau.B(s(x))$,
 $A(x)$和$A'(x)$是不符号互模拟的,
 因为不存在$x\geq 3$的划分,
 使得划分中的每一项推出$x\geq 5$的一个划分中的每一项。
 但是$(x\geq 5) A(x)$和$(x\geq 5) A'(x)$却是互模拟的,
 此时$A(x)=(x\geq 5)\tau.B(s(x)) = ((x\geq 5)\wedge (x\geq 3))\tau.B(s(x)) = (x\geq 5)A'(x)$,
 它们甚至是绝对等价的。
 在$[A]_{\varphi \mathcal{E}}$内部,
 我们可以使$\varphi$对等价性的影响透明。

\section{随机传值进程模型中的互模拟关系}

\subsection{互模拟关系与观察等价性}

   程序理论的基本问题是进程的等价性。对通信并发系统程序等价的定义和验证,现在已有很多研究。
   在提出CCS时Milner提出了\textit{观察等价(observation equivalence)},即\textit{弱互模拟(weak bisimulation)}\cite{2}。
   Glabbeek和Weijland提出的\textit{branching互模拟(branching bisimulation)}\cite{6}也是十分著名的研究。

   对于概率模型的等价性,目前有对同步概率进程模型互模拟关系的研究\cite{13},
   仅适用于有限状态的概率进程模型的研究\cite{14,15},
   这些模型都有一定的局限性。
   2019年,\cite{7}提出了一个模型无关的方法,在CCS的基础上对非概率模型进行概率化扩展,
   并在branching bisimulation\cite{6}的基础上给出了概率进程模型branching互模拟关系的语义,以及等价关系同余性的证明。

   \subsubsection{分支互模拟}

   \subsubsection{符号互模拟}
% \begin{definition}
%    对任意$S,T\in \mathcal{T}_{\mathbb{VPC}_{\mathsf{Th}}}$,且$V=fv(S|T)$,
%    当$S\mathcal{R}T$且满足下列条件时,对称二元关系$\mathcal{R}$是一个符号互模拟(Symbolic Bisimulation)[此处应有引用]。
%    \begin{itemize}
%       \item [1.] 
%    \end{itemize}
% \end{definition}
\subsection{条件Epsilon树}
\begin{definition}
   \label{def:silent_tree}
   若$A\in \mathcal{T}_{RVPC}$,
   $A$的静态转移树$t$满足如下定义:
   \begin{itemize}
   \item 每一个节点都被标记成$\mathcal{T}_{RVPC}$的一个元素,$A$是根节点。
   \item {
      节点间的边被标记成$(\varphi,p)$,其中$p\in(0,1]$,$\varphi$是一个布尔表达式。
      如果一条从$A'$到$A''$的有向边被标记成$(\varphi,p)$,表示$A'\stackrel{p\tau}{\rightarrow}_{\varphi} A''$。
   }
   \end{itemize}
\end{definition}

\begin{definition}[条件Epsilon树:$\varphi \mathcal{E}$-tree]
   $A$的静态转移树是一个条件Epsilon树($\varphi \mathcal{E}$-tree) $t^A_{\varphi \mathcal{E}}$ 
当且仅当下列条件成立
\begin{itemize}
   \item 树中的所有结点属于$[A]_{\varphi \mathcal{E}}$。
   \item 树中标记为$(\psi, q),q\in(0,1],V=fv_{\psi\wedge\varphi}$的边满足存在$V$的一组赋值$v'$使得$\mathsf{Th}\vdash \psi_{v'}\wedge\varphi_{v'}$,在$\varphi\mathcal{E}$-tree中,该边被重新标记为$(\varphi\psi,q)$。
\end{itemize}
\end{definition}

\begin{corollary}
   $A$的静态转移树是一个$\varphi \mathcal{E}$-tree $t^A_{\varphi \mathcal{E}}$ 则对于树上的节点$B,B',B_i,i\in [k]$:
   \begin{itemize}
      \item {
         $B\stackrel{(\psi,q)}{\rightarrow}B',q\in(0,1]$,则$\mathsf{Th}\vdash\psi\Rightarrow\varphi$。
      }
      \item {
         若$B\stackrel{(\psi,q)}{\rightarrow}B',q\in (0,1)$,则存在$B\stackrel{\coprod_{i\in [k]}}{\longrightarrow}_{\psi} \coprod_{i\in [k]} B_i$,
         且$B$有且仅有$B_1,\dots, B_k$这$k$个儿子节点。
      }
      \item {
         若存在$B\stackrel{(\psi,1)}{\rightarrow}B'$,则$B\stackrel{\tau}{\rightarrow}_{\psi}B'$且$B$有且仅有$B'$一个儿子节点。
      }
   \end{itemize}
\end{corollary}

条件Epsilon树的定义比较抽象,我们可以看几个简单有趣的例子:
\begin{example}
   $H(x)\stackrel{def}{=}(x\leq 3)(\frac{1}{3}\tau.(x\geq 1)G(x)\oplus\frac{1}{3}\tau.(x\geq 2)G(x)\oplus\frac{1}{3}\tau.(x\geq 3)G(x))$

   $[H(x)]_{\mathcal{E}_1} = [G(x)]_{\mathcal{E}_1}$。

   $H(x)$的静态转移树如图~\ref{fig_eg0_1}:
   \begin{figure}[!htbp]
      \caption[]{}
      \small
      \centering
      \includegraphics[width=8cm]{../figure/example0_1.png}
       \label{fig_eg0_1}
   \end{figure}

   $H(x)$的$\top\mathcal{E}_1$-tree只有一个根节点$H(x)$。

   $H(x)$的$(x>3)\mathcal{E}_1$-tree只有一个根节点$H(x)$。

   $H(x)$的$(x>2)\mathcal{E}_1$-tree如图~\ref{fig_eg0_2}:
   \begin{figure}[!htbp]
      \caption[]{}
      \small
      \centering
      \includegraphics[width=5cm]{../figure/example0_2.png}
       \label{fig_eg0_2}
   \end{figure}
\end{example}
\begin{example}\label{eg:1}
      假设我们有一个不太行的下课铃系统,每一时刻它坏掉的可能性是$\frac{1}{2}$,可以通过内部自动校准系统($\tau$操作)修复,它的内部有一个计时器,每隔定长时间($\tau$操作)就会从$a$通道广播录好的一段下课铃。
      这个下课铃系统可以被抽象为一个Random VPC,
      我们可以用$G(x)$来表示这个系统:
      $$G(x)\stackrel{def}{=}\mu X.(\frac{1}{2}\tau.X\oplus \frac{1}{2}\tau.\overline{a}(x).X)$$
      $G(x)$在$\mathcal{E}_2$的等价集$[G(x)]_{\top\mathcal{E}_2} = [\overline{a}(x).G(x)]_{\top\mathcal{E}_2}$。
      $G(x)$的$\top \mathcal{E}_2$-tree如图~\ref{fig_eg1}:
      \begin{figure}[!htbp]
         \caption[]{}
         \small
         \centering
         \includegraphics[width=5cm]{../figure/example1.png}
          \label{fig_eg1}
      \end{figure}
\end{example}
\begin{example}\label{eg:2}
   假设之前的不太行的下课铃系统经过岁月的磨练,年久失修,
   每自动校准一次音量就会减弱,音量必须满足$y>1$才可以播放,
   当$y\leq 0$时下课铃系统音量就无法减弱了。
   为了方便建模,我们用$\mathsf{p}(x)=x-1$来表示这种音量减弱。
   校长出于节约经费的考虑,只要下课铃还能在他办公室听见($y>3$)就可以继续使用,
   现在的下课铃系统依然是一个Random VPC,
   我们可以用修改的$G'(x,y)$来表示这个系统:
   $$G'(x,y)\stackrel{def}{=}(y>3)(\frac{1}{2}\tau.((y>0)\tau.G'(x,\mathsf{p}(y))|\urcorner (y>0)\tau.G'(x,y))\oplus \frac{1}{2}\tau.((y>1)\overline{a}(x).G'(x,y)))$$
   $G'(x,y)$在$(y>3)\mathcal{E}_3$的等价集$[G'(x,y)]_{(y>3)\mathcal{E_3}}=[G'(x,y')]_{(y'>3)\mathcal{E_3}}=[a(x).G'(x,y'')]_{(y''>3)\mathcal{E_3}}$,其中$y,y',y''$不一定相等。

   $G'(x,y)$的$(y>3)\mathcal{E}_3$-tree如图~\ref{fig_eg2}
   \begin{figure}[!htbp]
      \caption[]{}
      \small
      \centering
      \includegraphics[width=11cm]{../figure/example2.png}
       \label{fig_eg2}
   \end{figure}

\end{example}

\subsection{随机传值进程模型的符号互模拟}
\begin{definition}[条件概率转移:$\varphi q$-transition]
   $A\in \mathcal{T}_{RVPC},\mathcal{B}\in (\mathcal{T}_{RVPC}/\varphi \mathcal{E})\backslash \{[A]_{\varphi\mathcal{E}}\}$,
对$t^A_{\varphi \mathcal{E}}$的每一个叶子结点$L\stackrel{\coprod_{i\in I}p_i\tau}{\longrightarrow}_{\coprod_{i\in I}\psi_i} \coprod_{i\in [k]}L_i$,
$i\in I, L_i\in \mathcal{B}$且$\mathsf{Th}\vdash \varphi \Rightarrow \psi_i$。

定义$\mathsf{P}_\varphi(L\stackrel{\coprod_{i\in I}p_i\tau}{\longrightarrow}_{\coprod_{i\in I}\psi_i}\mathcal{B}) = \sum\{p_i\mid L\stackrel{p_i\tau}{\rightarrow}_{\psi_i} L_i\in\mathcal{B} \wedge i\in I \wedge \mathsf{Th}\vdash \varphi \Rightarrow \psi_i\}$。

定义$\mathsf{P}_{\varphi \mathcal{E}}(L\stackrel{\coprod_{i\in I}p_i\tau}{\longrightarrow}_{\coprod_{i\in I}\psi_i}\mathcal{B}) = \mathsf{P}_\varphi(L\stackrel{\coprod_{i\in I}p_i\tau}{\longrightarrow}_{\coprod_{i\in I}\psi_i}\mathcal{B})/(1-\mathsf{P}_\varphi(L\stackrel{\coprod_{i\in I}p_i\tau}{\longrightarrow}_{\coprod_{i\in I}\psi_i}[A]_{\varphi\mathcal{E}}))$。

当$\mathsf{P}_{\varphi \mathcal{E}}(L\stackrel{\coprod_{i\in I}p_i\tau}{\longrightarrow}_{\coprod_{i\in I}\psi_i}\mathcal{B})=q$时,
$A$到$\mathcal{B}$存在$\varphi q$-transition。写作$A\rightsquigarrow_{\varphi\mathcal{E}} \stackrel{q}{\rightarrow}_{\varphi} \mathcal{B}$。
\end{definition}


\begin{definition}[符号互模拟]
   $\mathcal{E}$是一个$\mathcal{T}_{RVPC}$上的二元对称关系,
当$A\mathcal{E}B$且满足下列条件时,称$\mathcal{E}$是一个符号互模拟(Symbolic Bisimulation):
\begin{itemize}
   \item {
      若$ A \rightsquigarrow_{\varphi \mathcal{E}}\stackrel{a(x)}{\rightarrow}_{\varphi} \mathcal{C}\in \mathcal{T}/\varphi\mathcal{E}, \mathcal{C}\neq [A]_{\mathcal{E}}$,
      则存在$\varphi$的划分$\{\varphi_i\}_{i\in I}$,和集合$\{B\rightsquigarrow_{\varphi_i \mathcal{E}}\stackrel{a(x)}{\rightarrow}_{\psi_i} \mathcal{C}\}$,
      使得对$i\in I$,$\mathsf{Th}\vdash \varphi_i \Rightarrow \psi_i$。
   }
   \item {
      若$A \rightsquigarrow_{\varphi \mathcal{E}}\stackrel{\bar{a}(t)}{\rightarrow}_{\varphi} \mathcal{C}\in \mathcal{T}/\varphi\mathcal{E}$,
      则存在$\varphi$的划分$\{\varphi_i\}_{i\in I}$,和集合$\{B\rightsquigarrow_{\varphi_i\mathcal{E}}\stackrel{\bar{a}(t_i)}{\rightarrow}_{\psi_i} \mathcal{C}\}$,
      使得对$i\in I$,$\mathsf{Th}\vdash (\varphi_i \Rightarrow \psi_i)\wedge (t=t_i)$。
   }
   \item {
      若$ A\rightsquigarrow_{\varphi\mathcal{E}} \stackrel{q}{\rightarrow}_{\varphi} \mathcal{C}\in \mathcal{T}/\mathcal{E}$,
      则$B\rightsquigarrow_{\varphi \mathcal{E}} \stackrel{q}{\rightarrow}_{\varphi} \mathcal{C}$
   }
\end{itemize}

\end{definition}

\begin{example}
   假设学校负责看管设备的老师发现例~\ref{eg:2}中的下课铃系统其实只有在$y\geq 5$时才能被所有教室的同学们听到,
   为了不违抗校长的决定又同时让同学们都听到下课铃,他决定当$(y<5)$时用自己的电脑通过$a$通道播放下课铃,
   这时校长以为下课铃系统被工人修好成~\ref{eg:1}中的样子了,决定一直使用这个下课铃。校长的感觉是错觉吗?

   此时,看管设备的老师和之前不太行的下课铃系统依旧是一个Random VPC,我们可以用$G''(x,y)$来表示:

   $G''(x,y) = (\frac{1}{2}\tau.((y>0)\tau.G''(x,\mathsf{p}(y))|\urcorner (y>0)\tau.G''(x,y))\oplus \frac{1}{2}\tau.((y>1)\overline{a}(x).G''(x,y)|(y<5)\overline{a}(x).G''(x,y))$。

   我们只需要证明$G(x)$与$G''(x,y)$符号互模拟即可。
\end{example}
\begin{proof}
   设等价集$\mathcal{S} = \{(G(t),G''(t,s)),(G(t),(y>0)\tau.G''(x,\mathsf{p}(y))|\urcorner (y>0)\tau.G''(x,y)),(\overline{a}.G(t),(s>1)\overline{a}(t).G''(t,s)|(s<5)\overline{a}(t).G''(t,s))\mid t\in V,s\in U\}$,$V$是$x$的所有赋值的集合,$U$是$y$的所有赋值的集合,
   我们只需证明$\mathcal{S}$是一个符号互模拟关系即可。
   \begin{itemize}
      \item {
         $\overline{a}(x).G(x)$和$(y>1)\overline{a}(x).G''(x,y)|(y<5)\overline{a}(x).G''(x,y)$的$\top\mathcal{S}$-tree只有一个根节点(这里的证明实际可以是$\mathbb{VPC}$的符号互模拟的证明,但可以使用Random VPC的证明方式,下面用Random VPC的证明),
         \begin{itemize}
            \item {
               对于$x$的每一个赋值$t$,
               $\overline{a}(x).G(x)\rightsquigarrow_{\top\mathcal{S}}\stackrel{\overline{a}(t)}{\longrightarrow}_{\top}G(t)$。
               我们可以得到$\top$的一个划分:$\top=(y>1)\vee (y\leq 1)$。
      
               这时存在集合
               $$\{(y>1)\overline{a}(x).G''(x,y)|(y<5)\overline{a}(x).G''(x,y)\rightsquigarrow_{(y>1)\mathcal{S}}\stackrel{\overline{a}(t)}{\rightarrow}_{(y>1)}G''(t,s),$$
               $$(y>1)\overline{a}(x).G''(x,y)|(y<5)\overline{a}(x).G''(x,y)\rightsquigarrow_{(y\leq 1)\mathcal{S}}\stackrel{\overline{a}(t)}{\rightarrow}_{(y<5)}G''(t,s)\}$$
               可以模拟上述操作。
            }
            \item {
               对$x$的每一个赋值$t$,
               $(y>1)\overline{a}(x).G''(x,y)|(y<5)\overline{a}(x).G''(x,y)\rightsquigarrow_{(y>1)\mathcal{S}}\stackrel{\overline{a}(t)}{\rightarrow}_{(y>1)}G''(t,s)$,
               存在集合$\{\overline{a}(x).G(x)\rightsquigarrow_{(y>1)\mathcal{S}}\stackrel{\overline{a}(t)}{\longrightarrow}_{(y>1)}G(t)\}$可以模拟上述操作。
               另一个操作的证明是相似的。
            }
         \end{itemize}
      }
      \item {
         $(G(t),(y>0)\tau.G''(x,\mathsf{p}(y))|\urcorner (y>0)\tau.G''(x,y))$的这一对等价关系也是常规的$\mathcal{VPC}$证明[此处应有引用]。
      }
      \item {
         $G(x)$的静态转移树如图~\ref{fig_eg4_1},$G''(x,y)$的静态转移树如图~\ref{fig_eg4_2}。
         \begin{figure}[!htbp]
            \caption[]{}
            \small
            \centering
            \includegraphics[width=5cm]{../figure/example1.png}
             \label{fig_eg4_1}
         \end{figure}
         \begin{figure}[!htbp]
            \caption[]{}
            \small
            \centering
            \includegraphics[width=13cm]{../figure/example4_2.png}
             \label{fig_eg4_2}
         \end{figure}

         对于$G(x)\rightsquigarrow_{\top\mathcal{S}}\stackrel{\frac{1}{2}}{\rightarrow}_{\top} [\overline{a}(x).G(x)]_{\top\mathcal{S}}$,
         我们可以用$G''(x,y)\rightsquigarrow_{\top\mathcal{S}}\stackrel{\frac{1}{2}}{\rightarrow}_{\top}[(y>1)\overline{a}(x).G''(x,y)|(y<5)\overline{a}(x).G''(x,y)]_{\top\mathcal{S}}$模拟。
         对称的,因为$\mathcal{S}$是一个等价关系,具有传递性,$(G(x),G''(x,y))\in\mathcal{S}\wedge (G(x),(y>0)\tau.G''(x,\mathsf{p}(y))|\urcorner (y>0)\tau.G''(x,y))\Rightarrow (y>0)\tau.G''(x,\mathsf{p}(y))|\urcorner (y>0)\tau.G''(x,y)\in [G(x)]_{\top\mathcal{S}}$,
         我们也只需要证明右侧的转移,右侧的转移是相似的。
      }
   \end{itemize}
   因此我们可以得出校长的判断是正确的。
\end{proof}

\begin{theorem}
   若$S,T\in \mathcal{T}_{RVPC},S\approxeq^s_{\mathsf{Th}} T$,则$\varphi S \approxeq^s_{\mathsf{Th}} \varphi T$,$\varphi$是一个判定条件。
\end{theorem}
该定理的证明在下一节。

\section{随机传值进程模型的等价性}

\begin{lemma}
   符号互模拟具有传递性。
\end{lemma} 
\begin{proof}
   即证明若$\mathcal{E}$是一个symbolic bisimulation,$A\mathcal{E}B, B \mathcal{E} C$,则$A \mathcal{E} C$。
   \begin{itemize}
      \item {
         若$A\rightsquigarrow_{\varphi \mathcal{E}}\stackrel{\lambda}{\rightarrow} \mathcal{C}\in \mathcal{T}_{RVPC}/\varphi\mathcal{E}$,
         由于$A\mathcal{E}B$根据定义存在$\varphi$的划分$\{\varphi_i\}$和集合$\{B\rightsquigarrow_{\varphi_i\mathcal{E}}\stackrel{\lambda}{\rightarrow}_{\psi}\mathcal{C}|\mathsf{Th}\vdash \varphi_i\Rightarrow\psi_i\}$。
         对于每一个$B\rightsquigarrow_{\varphi_i\mathcal{E}}\stackrel{\lambda}{\rightarrow}_{\psi_i}\mathcal{C}$,
         $t^B_{\varphi_i\mathcal{E}}$上的每一个结点一定属于$[B]_{\psi_i\mathcal{E}}$,
         所以$t^b_{\varphi_i\mathcal{E}}$实际上是$t^b_{\psi_i\mathcal{E}}$的子树,
         $B\rightsquigarrow_{\varphi_i\mathcal{E}}\stackrel{\lambda}{\rightarrow}_{\psi_i}\mathcal{C}$中的起点终点对$\{(M,N)|(M\in t^B_{\varphi_i\mathcal{E}})\vee(N\in \mathcal{C}) \vee (M\stackrel{\lambda}{\rightarrow}_{\psi_i}N))\}$是
         $B\rightsquigarrow_{\psi_i\mathcal{E}}\stackrel{\lambda}{\rightarrow}_{\psi_i}\mathcal{C}$中的起点终点对$\{(M',N')|(M'\in t^B_{\psi_i\mathcal{E}})\vee(N'\in \mathcal{C}) \vee (M'\stackrel{\lambda}{\rightarrow}_{\psi_i}N'))\}$的子集。
         由于$B\mathcal{E}C$,对于每一个$B\rightsquigarrow_{\psi_i}\stackrel{\lambda}{\rightarrow}_{\psi_i}\mathcal{C}$,
         根据定义存在$\psi_i$的划分$\{\psi_{i,j}\}$和集合$S=\{C\rightsquigarrow_{\psi_{i,j}\mathcal{E}}\stackrel{\lambda}{\rightarrow}_{\phi_{i,j}}\mathcal{C}|\mathsf{Th}\vdash \psi_{i,j}\Rightarrow \phi_{i,j}\}$,
         那么存在$S$的子集$S'$的可以模拟$B\rightsquigarrow_{\varphi_i\mathcal{E}}\stackrel{\lambda}{\rightarrow}_{\psi_i}\mathcal{C}$。
   
         我们只需要再构建$\varphi$的划分$\{\varphi_i'\}_{i\in I}$使其一一对应$S'$中涉及的条件$\{\phi_{i,k}\}_{k\in K}\subset\{\phi_{i,j}\}_{j\in J}$即可。
         由于存在$\psi_i$的划分$\{\psi_{i,j}\}_{j\in J}$满足$\psi_{i,j}\Rightarrow \phi_{i,j}\}_{j\in J}$,
         那么我们可以构造$Con=\{\psi_{i,k}\}_{k\in [K-1]}\cup\{\bigvee_{j\in J/[K-1]}\psi_{i,j}\}$,
         对$c_k\in Con, \bigvee_{k\in K}c_k \Leftrightarrow \psi_i$, 且$Con$中元素两两互斥。
         我们用$(\bigvee Con_i)\wedge \varphi_i$代替$\varphi_i$,即可得到最终的划分。
      }
      \item {
         $\varphi q$-transition的传递性与以上过程相似。
      }
   \end{itemize}
\end{proof}
\begin{lemma}
   如果每一个$\mathcal{T}_{RVPC}$上的关系$\mathcal{E}_i$都是符号互模拟,那么$(\bigcup_{i\in I}\mathcal{E}_i)^*$是符号互模拟。
\end{lemma}
\begin{proof}
   令$\mathcal{E}=(\bigcup_{i\in I}\mathcal{E}_i)^*$。若因为$\varphi A_0\mathcal{E}_1 \varphi A_1 \mathcal{E}_2 \dots \mathcal{E}_k \varphi A_k$ ,
$(\varphi A_0, \varphi A_k)\in \mathcal{E}$,由于互模拟具有传递性,通过依次证明$A_0\mathcal{E}A_1, A_1\mathcal{E}A_2 \dots$,我们可以证明$A_0$和$A_k$互模拟。

我们可以通过$A_0$的$\varphi\mathcal{E}$-tree,$t_{A_0}$递归的构建$A_1$的$\varphi\mathcal{E}$-tree,$t_{A_1}$,
进而对$A_0\rightsquigarrow_{\varphi\mathcal{E}}\stackrel{\lambda}{\rightarrow}_{\varphi}\mathcal{C}$构造出$\{A_1\rightsquigarrow_{\varphi_i\mathcal{E}}\stackrel{\lambda}{\rightarrow}_{\psi_i} \mathcal{C}|\mathsf{Th}\vdash (\varphi_i\Rightarrow\psi_i)\vee (\textrm{$\{\varphi_i\}$是$\varphi$的一个划分})\}$。
对于每次递归的$t_{A_0}$的根结点,分以下情况讨论:

\begin{itemize}
   \item {
      \textbf{Case 1 $t_{A_0}$的根结点只有一个儿子$A_0'$。}
      \begin{itemize}
         \item {
            \textbf{Case 1.1 $A_0'\in [A_0]_{\varphi\mathcal{E}_1}$。} 则根据$A_0'$构建$A_1$的$\varphi\mathcal{E}$-tree。
         }
         \item {
            \textbf{Case 1.2 $A_0'\notin [A_0]_{\varphi\mathcal{E}_1}$。} 
            根据定义存在划分$\{\varphi_i\}$
            和集合$\{A_1\rightsquigarrow_{\varphi_i\mathcal{E}}\stackrel{\lambda}{\rightarrow}_{\psi_i} \in [A_0']_{\varphi\mathcal{E}}|\mathsf{Th}\vdash \varphi_i\Rightarrow\psi_i\}$。
            这里我们构建了一个$A_1$的$\varphi\mathcal{E}_1$-tree, $t_{A_1}'$。对$t_{A_1}'$的叶子结点$B\stackrel{\lambda}{\rightarrow}_{\psi_i} B'$中的$B'\in[A_0']_{\varphi\mathcal{E}_1}$,
            根据$A_0'$的$\varphi\mathcal{E}$-tree构建$B'$的$\varphi\mathcal{E}$-tree。
         }
      \end{itemize}
   }
   \item {
      \textbf{Case 2 $A_0$有$h$个儿子$A_0^1,\dots, A_0^h$。}
      \begin{itemize}
         \item {
            \textbf{Case 2.1 $\forall j\in [h], A^j_0\mathcal{E}_1 A_0$。}我们根据$A^1_0$构建$t_{A_1}$。
         }
         \item {
            \textbf{Case 2.2 存在$A^1_0\notin[A_0]_{\varphi\mathcal{E}_1}$。}
            令$q = \mathsf{P}_{\varphi\mathcal{E}_1}(A_0\stackrel{\coprod_{i\in [h]p_i\tau}}{\rightarrow}_{\coprod_{i\in I}\psi_i}[A^1_0]_{\varphi\mathcal{E}_1})$,
            有$A_1\rightsquigarrow_{\varphi \mathcal{E}_1}\stackrel{q}{\rightarrow}_{\varphi} [A^1_0]_{\varphi\mathcal{E}_1}$。
            对于$A_1$的$\varphi\mathcal{E}_1$-tree,$t_{A_1}'$的叶子结点$N$,$N\stackrel{\coprod_{i\in [h]}p_i\tau}{\rightarrow}_{\coprod_{i\in I}\psi_i} \coprod_{i\in I}N_i'\in [A^1_0]_{\varphi\mathcal{E}_1}$中的$N_i'$,
            根据$A^1_0$来构造$N'_i$的$\varphi\mathcal{E}$-tree。
         }
      \end{itemize}
   }
   \item {
      \textbf{Case 3 $t_{A_0}$的根结点$A_0\stackrel{\lambda}{\rightarrow}_{\varphi} L'\in \mathcal{C}$。}
      根据定义存在划分$\{\varphi_i\}, \mathsf{Th}\vdash\bigvee_{i\in I}\varphi_i \Leftrightarrow \varphi$
      和集合$\{A_1\rightsquigarrow_{\varphi_i\mathcal{E}_1}\stackrel{\lambda}{\rightarrow}_{\psi_i} \in \mathcal{C}|\mathsf{Th}\vdash \varphi_i\Rightarrow\psi_i\}$,
      由于$\mathcal{E}_1\in (\bigcup_{i\in I}\mathcal{E}_i)^*$, 我们可以得到存在集合$\{A_1\rightsquigarrow_{\varphi_i\mathcal{E}}\stackrel{\lambda}{\rightarrow}_{\psi_i} \in \mathcal{C}|\mathsf{Th}\vdash \varphi_i\Rightarrow\psi_i\}$。
   }
\end{itemize}
\end{proof}

\begin{definition}
   $\mathcal{T}_{RVPC}$上的观察等价性$\approxeq^s_{\mathsf{Th}}$定义为$\mathsf{Th}$上的符号互模拟的全集。
\end{definition}

\begin{theorem}
   $\approxeq^s_{\mathsf{Th}}$是一个等价关系。
\end{theorem}

\begin{theorem}
   $\approxeq^s_{\mathsf{Th}}$具有同余性。
\end{theorem}
\begin{proof}
   $\approxeq^s_{\mathsf{Th}}$对于随机选择和非确定性选择的封闭性比较容易证明。

对条件操作运算的封闭性在于,$S\approxeq^s_{\mathsf{Th}} T$可以推出$\varphi S \approxeq^s_{\mathsf{Th}} \varphi T$
对$\varphi S\rightsquigarrow_{\psi \approxeq^s_{\mathsf{Th}}}\stackrel{\lambda}{\rightarrow}_{\psi} \mathcal{C}\in \mathcal{T}/\psi \approxeq^s_{\mathsf{Th}}$,
有$\mathsf{Th}\vdash \varphi\psi$,且$S\rightsquigarrow_{\psi \approxeq^s_{\mathsf{Th}}}\stackrel{\lambda}{\rightarrow}_{\psi} \mathcal{C}\in \mathcal{T}/\psi \approxeq^s_{\mathsf{Th}}$,
根据定义,存在$\psi$的划分$\{\psi_i\}_{i\in I}$和集合$\{T\rightsquigarrow_{\psi_i \approxeq^s_{\mathsf{Th}}}\stackrel{\lambda}{\rightarrow}_{\phi_i}\mathcal{C}|\mathsf{Th}\vdash \psi_i\Rightarrow\phi_i\}$,
又因为$\mathsf{Th}\vdash \varphi\psi$,所以存在$j\in J\subset I, \mathsf{Th}\vdash \varphi\psi_j$,$\{\varphi T\rightsquigarrow_{\psi_j \approxeq^s_{\mathsf{Th}}}\stackrel{\lambda}{\rightarrow}_{\phi_j}\mathcal{C}|\mathsf{Th}\vdash \psi_j\Rightarrow\phi_j\}$,
重新构造$\psi$的划分为$\{\psi_j\}_{j\in [J-1]}\cup \{\bigvee_{i\in [I]/[J-1]}\psi_i\}$。q-transition的证明是类似的。

Composition,Localization,Recursion的运算与$\mathbb{VPC}$的运算[此处应有引用]相同。
\end{proof}