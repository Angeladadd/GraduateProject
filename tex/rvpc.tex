% !TeX root = ../thesis.tex

\chapter{随机传值进程模型}

\section{Random VPC语法}
根据Uniform Approach, 我们将Uniform Approach中的CCS部分替换为VPC,很容易得到Random VPC的语法。
$$T:=\bigoplus_{i\in I}p_i \tau.T_i\mid \sum_{i\in I} \varphi_i\lambda_i.T_i\mid T \mid T'\mid (c)T\mid \varphi T\mid !a(x).T \mid !\bar{a}(t).T$$

Uniform Approach修改了CCS的prefix操作符,
在CCS的基础上增加了$p\tau.T$,使$\tau$操作成为一个随机性的操作。

在扩展VPC时,我们也将重点关注prefix操作符。

在VPC的Symbolic Semantics中,
对于VPC的典型prefix运算$\varphi \lambda. T\stackrel{\lambda}{\rightarrow}_{\varphi} T$,
我们可以通过Uniform Approach的方法将它扩展为一个概率的操作
$p\tau.\varphi \lambda. T\stackrel{p\tau}{\rightarrow}_{\top}\stackrel{\lambda}{\rightarrow}_{\varphi} T$。
它的语义就会变成:在概率$p$下,我们会经过一个内部$\tau$通道进入一个VPC $\varphi\lambda.T$,
若$\mathsf{Th}\vdash \varphi$(即一阶理论$\mathsf{Th}$下,$\varphi$为真),
则我们可以经过$\lambda$通道到达$T$, 其中$\lambda \in \{a(x),\bar{a}(x)\mid a\in \mathcal{N}, x\in \mathsf{V}_\Sigma, t\in \mathsf{T}_\Sigma\}\cup \{\tau\}$。

\subsection{if then else语法}

这个语法在VPC和CCS里看起来都非常格格不入,我们尝试用VPC的语法来表示所有的if then else,并给出相应的证明。

在VPC中,$\varphi T \equiv if\; \varphi \; then \; T$, $(\varphi T\mid \urcorner \varphi S)\equiv if \; \varphi \; then\; T \; else \; S$。
那么$\varphi T \stackrel{\lambda}{\rightarrow}_{\urcorner \varphi} ?$,$\varphi T$是否等价于$(\varphi T\mid \urcorner \varphi 0)$?

\textbf{Proposition 1} $\varphi 0 = 0$

\textit{Proof.} $\varphi 0 = \varphi 0 + 0 = \varphi 0 + \top 0 = \varphi 0 + (\varphi \vee \urcorner \varphi)0 = \varphi 0 + \varphi 0 + \urcorner \varphi 0 = \varphi 0 + \urcorner \varphi 0 = (\varphi \vee \urcorner \varphi)0 = \top 0 = 0$

\textbf{Proposition 2} $\varphi T = (\varphi T\mid \urcorner \varphi 0)$

\textit{Proof.} $\varphi T = (\varphi T\mid 0) = (\varphi T\mid \urcorner \varphi 0)$ according to Proposition 1.

\textbf{Proposition 3} $S=\varphi T$,那么$S\stackrel{\epsilon}{\rightarrow}_{\urcorner \varphi} 0$

\subsection{操作子的定义与条件等价集}

对于$S = p\tau.\varphi T+(1-p)\tau.0$,它的意义是在$p$的概率下选择内部通道$\tau$,
若$\mathsf{Th}\vdash\varphi$,则到达$T$状态。
在VPC的Symbolic Semantic下,它会经过这样的转移:$S \stackrel{p\tau}{\rightarrow} \stackrel{\epsilon}{\rightarrow}_{\varphi} T$。
考虑是否可以将其写成这种形式: $S\stackrel{p}{\Rightarrow}_{\varphi} T$。

\textbf{Definition 1} 若$S,T\in \mathcal{T}_{\mathbb{VPC}}$,$S\stackrel{\tau}{\rightarrow}_{\varphi_1}\stackrel{\tau}{\rightarrow}_{\varphi_2}\dots \stackrel{\tau}{\rightarrow}_{\varphi_n} T$,则可以记成$S\stackrel{}{\Rightarrow}_{\varphi_1\varphi_2\dots\varphi_n} T$。

\textbf{Definition 2} 若$S,T\in \mathcal{P}_{RCCS}$,$S\stackrel{p_1\tau}{\rightarrow}\stackrel{p_2\tau}{\rightarrow}\dots\stackrel{p_n\tau}{\rightarrow}T$,则记为$S\stackrel{p_1p_2\dots p_n}{\Rightarrow}T$。

以上,定义1在VPC中有相同的定义,
而Uniform Approach中没有类似定义2的定义,
但在方便表达的情境下,我们新增了这样的记号。

\textbf{Definition 3} 若$S,T\in \mathcal{T}_{RVPC}$,
$S\stackrel{p_1\tau}{\rightarrow}_{\varphi_1}\stackrel{p_2\tau}{\rightarrow}_{\varphi_2}\dots \stackrel{p_n\tau}{\rightarrow}_{\varphi_n} T$,则可以记成$S\stackrel{p_1p_2\dots p_n}{\Rightarrow}_{\varphi_1\varphi_2\dots\varphi_n} T$。
若$\varphi = \varphi_1\varphi_2\dots \varphi_n, p=p_1p_2\dots p_n$,
则$S\stackrel{p}{\Rightarrow}_{\varphi} T$。

\textbf{Definition 4} 若$S=\varphi p\tau.T$,那么$S\stackrel{p\tau}{\rightarrow}_{\varphi} T$

 \textbf{Definition 5} $A,A'\in \mathcal{T}_{RVPC}$,若$\varphi A \mathcal{E} \varphi A'$,则$A'\in [A]_{\varphi \mathcal{E}}$。
 $[A]_{\varphi \mathcal{E}}$称为等价关系$\mathcal{E}$在条件$\varphi$下包含$A$的等价集。

 定义5的产生是由于,在symbolic bisimulation中,
 我们考虑用必要条件下的动作去模拟充分条件下的动作。
 举个简单的例子
 对于$A(x)=(x\geq 5)\tau.B(s(x))$
 和$A'(x)=(x\geq 3)\tau.B(s(x))$,
 $A(x)$和$A'(x)$是不符号互模拟的,
 因为不存在$x\geq 3$的划分,
 使得划分中的每一项推出$x\geq 5$的一个划分中的每一项。
 但是$(x\geq 5) A(x)$和$(x\geq 5) A'(x)$却是互模拟的。
 在$[A]_{\varphi \mathcal{E}}$内部,
 我们可以使$\varphi$对等价性的影响透明。

\section{Epsilon Tree}

\textbf{Definition 6} smybolic bisimulation在$\mathcal{T}_{VPC}$上的定义 VPC Page 181(此处略)

\textbf{Definition 7} 若$A\in \mathcal{T}_{RVPC}$,
$A$的silent tree t满足如下定义:
\begin{itemize}
   \item 每一个节点都被标记成$\mathcal{T}_{RVPC}$的一个元素,$A$是根节点。
   \item {
      边被标记成$(\varphi,p)$,其中$p\in(0,1]$,$\varphi$是一个布尔表达式。
      如果一条从$A'$到$A''$的有向边被标记成$(\varphi,p)$,表示$A'\stackrel{p\tau}{\rightarrow}_{\varphi} A''$。
      特别的,如果存在$A'\stackrel{\tau}{\rightarrow}_{\varphi}A''$,
      那么$A'$到$A''$的边标记为$(\varphi, 1)$;存在$A'\stackrel{p\tau}{\rightarrow}A''$,
      $A'$到$A''$的边标记为$(\top, 1)$。
   }
\end{itemize}

\textbf{Definition 8} $A$的silent tree是一个$\varphi \mathcal{E}-tree\; t^A_{\varphi \mathcal{E}}$ 
当且仅当树中的所有结点属于$[A]_{\varphi \mathcal{E}}$。

\textbf{Definition 9} $\varphi q$-transition。
$A\in \mathcal{T}_{RVPC},\mathcal{B}\in (\mathcal{T}_{RVPC}/\varphi \mathcal{E})\backslash \{[A]_{\varphi\mathcal{E}}\}$,
对$t^A_{\varphi \mathcal{E}}$的每一个叶子结点$L\stackrel{\coprod_{i\in I}p_i\tau}{\longrightarrow}_{\coprod_{i\in I}\psi_i} \coprod_{i\in [k]}L_i$,
存在$i\in I, L_i\in \mathcal{B}$且$\mathsf{Th}\vdash \varphi \Rightarrow \psi_i$。
定义$\mathsf{P}_\varphi(L\stackrel{\coprod_{i\in I}p_i\tau}{\longrightarrow}_{\coprod_{i\in I}\psi_i}\mathcal{B}) = \sum\{p_i\mid L\stackrel{p_i\tau}{\rightarrow}_{\psi_i} L_i\in\mathcal{B} \wedge i\in I \wedge \mathsf{Th}\vdash \varphi \Rightarrow \psi_i\}$。
定义$\mathsf{P}_{\varphi \mathcal{E}}(L\stackrel{\coprod_{i\in I}p_i\tau}{\longrightarrow}_{\coprod_{i\in I}\psi_i}\mathcal{B}) = \mathsf{P}_\varphi(L\stackrel{\coprod_{i\in I}p_i\tau}{\longrightarrow}_{\coprod_{i\in I}\psi_i}\mathcal{B})/(1-\mathsf{P}_\varphi(L\stackrel{\coprod_{i\in I}p_i\tau}{\longrightarrow}_{\coprod_{i\in I}\psi_i}[A]_{\varphi\mathcal{E}}))$。
当$\mathsf{P}_{\varphi \mathcal{E}}(L\stackrel{\coprod_{i\in I}p_i\tau}{\longrightarrow}_{\coprod_{i\in I}\psi_i}\mathcal{B})=q$时,
$A$到$\mathcal{B}$有$\varphi q$-transition。
写作$A\rightsquigarrow_{\varphi\mathcal{E}} \stackrel{q}{\rightarrow}_{\varphi} \mathcal{B}$。


\textbf{Definition 10} 一个$\mathcal{T}_{RVPC}$上的等价关系$\mathcal{E}$是一个symbolic bisimulation,
当且仅当, $A\mathcal{E}B$则($A,B$对称)
\begin{itemize}
   \item {
      若$ A \rightsquigarrow_{\varphi \mathcal{E}}\stackrel{a(x)}{\rightarrow}_{\varphi} \mathcal{C}\in \mathcal{T}/\varphi\mathcal{E}, \mathcal{C}\neq [A]_{\mathcal{E}}$,
      则存在$\varphi$的划分$\{\varphi_i\}_{i\in I}$,和集合$\{B\rightsquigarrow_{\varphi_i \mathcal{E}}\stackrel{a(x)}{\rightarrow}_{\psi_i} \mathcal{C}\}$,
      使得对$i\in I$,$\mathsf{Th}\vdash \varphi_i \Rightarrow \psi_i$。
   }
   \item {
      若$A \rightsquigarrow_{\varphi \mathcal{E}}\stackrel{\bar{a}(t)}{\rightarrow}_{\varphi} \mathcal{C}\in \mathcal{T}/\varphi\mathcal{E}$,
      则存在$\varphi$的划分$\{\varphi_i\}_{i\in I}$,和集合$\{B\rightsquigarrow_{\varphi_i\mathcal{E}}\stackrel{\bar{a}(t_i)}{\rightarrow}_{\psi_i} \mathcal{C}\}$,
      使得对$i\in I$,$\mathsf{Th}\vdash (\varphi_i \Rightarrow \psi_i)\wedge (t=t_i)$。
   }
   \item {
      若$ A\rightsquigarrow_{\varphi\mathcal{E}} \stackrel{q}{\rightarrow}_{\varphi} \mathcal{C}\in \mathcal{T}/\mathcal{E}$,
      则$B\rightsquigarrow_{\varphi \mathcal{E}} \stackrel{q}{\rightarrow}_{\varphi} \mathcal{C}$
   }
\end{itemize}

\section{Random VPC的等价性}

\textbf{Proposition 4} symbolic bisimulation具有传递性。

\textit{Proof.} 即证明若$\mathcal{E}$是一个symbolic bisimulation,$A\mathcal{E}B, B \mathcal{E} C$,则$A \mathcal{E} C$。
\begin{itemize}
   \item {
      若$A\rightsquigarrow_{\varphi \mathcal{E}}\stackrel{\lambda}{\rightarrow} \mathcal{C}\in \mathcal{T}_{RVPC}/\varphi\mathcal{E}$,
      由于$A\mathcal{E}B$根据定义存在$\varphi$的划分$\{\varphi_i\}$和集合$\{B\rightsquigarrow_{\varphi_i\mathcal{E}}\stackrel{\lambda}{\rightarrow}_{\psi}\mathcal{C}|\mathsf{Th}\vdash \varphi_i\Rightarrow\psi_i\}$。
      对于每一个$B\rightsquigarrow_{\varphi_i\mathcal{E}}\stackrel{\lambda}{\rightarrow}_{\psi_i}\mathcal{C}$,
      $t^B_{\varphi_i\mathcal{E}}$上的每一个结点一定属于$[B]_{\psi_i\mathcal{E}}$,
      所以$t^b_{\varphi_i\mathcal{E}}$实际上是$t^b_{\psi_i\mathcal{E}}$的子树,
      $B\rightsquigarrow_{\varphi_i\mathcal{E}}\stackrel{\lambda}{\rightarrow}_{\psi_i}\mathcal{C}$中的起点终点对$\{(M,N)|(M\in t^B_{\varphi_i\mathcal{E}})\vee(N\in \mathcal{C}) \vee (M\stackrel{\lambda}{\rightarrow}_{\psi_i}N))\}$是
      $B\rightsquigarrow_{\psi_i\mathcal{E}}\stackrel{\lambda}{\rightarrow}_{\psi_i}\mathcal{C}$中的起点终点对$\{(M',N')|(M'\in t^B_{\psi_i\mathcal{E}})\vee(N'\in \mathcal{C}) \vee (M'\stackrel{\lambda}{\rightarrow}_{\psi_i}N'))\}$的子集。
      由于$B\mathcal{E}C$,对于每一个$B\rightsquigarrow_{\psi_i}\stackrel{\lambda}{\rightarrow}_{\psi_i}\mathcal{C}$,
      根据定义存在$\psi_i$的划分$\{\psi_{i,j}\}$和集合$S=\{C\rightsquigarrow_{\psi_{i,j}\mathcal{E}}\stackrel{\lambda}{\rightarrow}_{\phi_{i,j}}\mathcal{C}|\mathsf{Th}\vdash \psi_{i,j}\Rightarrow \phi_{i,j}\}$,
      那么存在$S$的子集$S'$的可以模拟$B\rightsquigarrow_{\varphi_i\mathcal{E}}\stackrel{\lambda}{\rightarrow}_{\psi_i}\mathcal{C}$。

      我们只需要再构建$\varphi$的划分$\{\varphi_i'\}_{i\in I}$使其一一对应$S'$中涉及的条件$\{\phi_{i,k}\}_{k\in K}\subset\{\phi_{i,j}\}_{j\in J}$即可。
      由于存在$\psi_i$的划分$\{\psi_{i,j}\}_{j\in J}$满足$\psi_{i,j}\Rightarrow \phi_{i,j}\}_{j\in J}$,
      那么我们可以构造$Con=\{\psi_{i,k}\}_{k\in [K-1]}\cup\{\bigvee_{j\in J/[K-1]}\psi_{i,j}\}$,
      对$c_k\in Con, \bigvee_{k\in K}c_k \Leftrightarrow \psi_i$, 且$Con$中元素两两互斥。
      进而,我们用$(\bigvee Con_i)\wedge \varphi_i$代替$\varphi_i$,即可得到最终的划分。
   }
   \item {
      $\varphi q$-transition的传递性与以上过程相似。
   }
\end{itemize}

\textbf{Proposition 5} 如果每一个$\mathcal{T}_{RVPC}$上的等价关系$\mathcal{E}_i$都是symbolic bisimulation,那么$(\bigcup_{i\in I}\mathcal{E}_i)^*$是一个symbolic bisimulation。

\textit{Proof.} 令$\mathcal{E}=(\bigcup_{i\in I}\mathcal{E}_i)$。若因为$\varphi A_0\mathcal{E}_1 \varphi A_1 \mathcal{E}_2 \dots \mathcal{E}_k \varphi A_k$ ,
$(\varphi A_0, \varphi A_k)\in \mathcal{E}$,由于互模拟具有传递性,通过依次证明$A_0\mathcal{E}A_1, A_1\mathcal{E}A_2 \dots$,我们可以证明$A_0$和$A_k$互模拟。

我们可以通过$A_0$的$\varphi\mathcal{E}$-tree,$t_{A_0}$递归的构建$A_1$的$\varphi\mathcal{E}$-tree,$t_{A_1}$,
进而对$A_0\rightsquigarrow_{\varphi\mathcal{E}}\stackrel{\lambda}{\rightarrow}_{\varphi}\mathcal{C}$构造出$\{A_1\rightsquigarrow_{\varphi_i\mathcal{E}}\stackrel{\lambda}{\rightarrow}_{\psi_i} \mathcal{C}|\mathsf{Th}\vdash (\varphi_i\Rightarrow\psi_i)\vee (\textrm{$\{\varphi_i\}$是$\varphi$的一个划分})\}$。
对于每次递归的$t_{A_0}$的根结点,分以下情况讨论:

\begin{itemize}
   \item {
      \textbf{Case 1 $t_{A_0}$的根结点只有一个儿子$A_0'$。}
      \begin{itemize}
         \item {
            \textbf{Case 1.1 $A_0'\in [A_0]_{\varphi\mathcal{E}_1}$。} 则根据$A_0'$构建$A_1$的$\varphi\mathcal{E}$-tree。
         }
         \item {
            \textbf{Case 1.2 $A_0'\notin [A_0]_{\varphi\mathcal{E}_1}$。} 
            根据定义存在划分$\{\varphi_i\}$
            和集合$\{A_1\rightsquigarrow_{\varphi_i\mathcal{E}}\stackrel{\lambda}{\rightarrow}_{\psi_i} \in [A_0']_{\varphi\mathcal{E}}|\mathsf{Th}\vdash \varphi_i\Rightarrow\psi_i\}$。
            这里我们构建了一个$A_1$的$\varphi\mathcal{E}_1$-tree, $t_{A_1}'$。对$t_{A_1}'$的叶子结点$B\stackrel{\lambda}{\rightarrow}_{\psi_i} B'$中的$B'\in[A_0']_{\varphi\mathcal{E}_1}$,
            根据$A_0'$的$\varphi\mathcal{E}$-tree构建$B'$的$\varphi\mathcal{E}$-tree。
         }
      \end{itemize}
   }
   \item {
      \textbf{Case 2 $A_0$有$h$个儿子$A_0^1,\dots, A_0^h$。}
      \begin{itemize}
         \item {
            \textbf{Case 2.1 $\forall j\in [h], A^j_0\mathcal{E}_1 A_0$。}我们根据$A^1_0$构建$t_{A_1}$。
         }
         \item {
            \textbf{Case 2.2 存在$A^1_0\notin[A_0]_{\varphi\mathcal{E}_1}$。}
            令$q = \mathsf{P}_{\varphi\mathcal{E}_1}(A_0\stackrel{\coprod_{i\in [h]p_i\tau}}{\rightarrow}_{\coprod_{i\in I}\psi_i}[A^1_0]_{\varphi\mathcal{E}_1})$,
            有$A_1\rightsquigarrow_{\varphi \mathcal{E}_1}\stackrel{q}{\rightarrow}_{\varphi} [A^1_0]_{\varphi\mathcal{E}_1}$。
            对于$A_1$的$\varphi\mathcal{E}_1$-tree,$t_{A_1}'$的叶子结点$N$,$N\stackrel{\coprod_{i\in [h]}p_i\tau}{\rightarrow}_{\coprod_{i\in I}\psi_i} \coprod_{i\in I}N_i'\in [A^1_0]_{\varphi\mathcal{E}_1}$中的$N_i'$,
            根据$A^1_0$来构造$N'_i$的$\varphi\mathcal{E}$-tree。
         }
      \end{itemize}
   }
   \item {
      \textbf{Case 3 $t_{A_0}$的根结点$A_0\stackrel{\lambda}{\rightarrow}_{\varphi} L'\in \mathcal{C}$。}
      根据定义存在划分$\{\varphi_i\}, \mathsf{Th}\vdash\bigvee_{i\in I}\varphi_i \Leftrightarrow \varphi$
      和集合$\{A_1\rightsquigarrow_{\varphi_i\mathcal{E}_1}\stackrel{\lambda}{\rightarrow}_{\psi_i} \in \mathcal{C}|\mathsf{Th}\vdash \varphi_i\Rightarrow\psi_i\}$,
      由于$\mathcal{E}_1\in (\bigcup_{i\in I}\mathcal{E}_i)^*$, 我们可以得到存在集合$\{A_1\rightsquigarrow_{\varphi_i\mathcal{E}}\stackrel{\lambda}{\rightarrow}_{\psi_i} \in \mathcal{C}|\mathsf{Th}\vdash \varphi_i\Rightarrow\psi_i\}$。
   }
\end{itemize}

\textbf{Definition 11} 如果$\mathcal{E}_i$是一个$\mathcal{T}_{RVPC}$上的symbolic bisimulation,那么观察等价性$\approxeq^s_{\mathsf{Th}}$定义等价为最大的symbolic bisimulation。

以下$\approxeq^s_{\mathsf{Th}}$均为$\mathcal{T}_{RVPC}$上的symbolic bisimulation。

\textbf{Proposition 5} $\approxeq^s_{\mathsf{Th}}$是一个等价关系。

\textbf{Proposition 6} $\approxeq^s_{\mathsf{Th}}$具有同余性。

\textit{Proof.} $\approxeq^s_{\mathsf{Th}}$对于随机选择和非确定性选择的封闭性比较容易证明。

对条件操作运算的封闭性在于,$S\approxeq^s_{\mathsf{Th}} T$可以推出$\varphi S \approxeq^s_{\mathsf{Th}} \varphi T$
对$\varphi S\rightsquigarrow_{\psi \approxeq^s_{\mathsf{Th}}}\stackrel{\lambda}{\rightarrow}_{\psi} \mathcal{C}\in \mathcal{T}/\psi \approxeq^s_{\mathsf{Th}}$,
有$\mathsf{Th}\vdash \varphi\psi$,且$S\rightsquigarrow_{\psi \approxeq^s_{\mathsf{Th}}}\stackrel{\lambda}{\rightarrow}_{\psi} \mathcal{C}\in \mathcal{T}/\psi \approxeq^s_{\mathsf{Th}}$,
根据定义,存在$\psi$的划分$\{\psi_i\}_{i\in I}$和集合$\{T\rightsquigarrow_{\psi_i \approxeq^s_{\mathsf{Th}}}\stackrel{\lambda}{\rightarrow}_{\phi_i}\mathcal{C}|\mathsf{Th}\vdash \psi_i\Rightarrow\phi_i\}$,
又因为$\mathsf{Th}\vdash \varphi\psi$,所以存在$j\in J\subset I, \mathsf{Th}\vdash \varphi\psi_j$,$\{\varphi T\rightsquigarrow_{\psi_j \approxeq^s_{\mathsf{Th}}}\stackrel{\lambda}{\rightarrow}_{\phi_j}\mathcal{C}|\mathsf{Th}\vdash \psi_j\Rightarrow\phi_j\}$,
重新构造$\psi$的划分为$\{\psi_j\}_{j\in [J-1]}\cup \{\bigvee_{i\in [I]/[J-1]}\psi_i\}$。q-transition的证明是类似的。

Composition和Localization的证明与Uniform Approach中的基本类似,
因为Random VPC没有对CCS的Composition和Localization做改动。

(由于时间原因这里证明的比较草率,中期检查之后会继续完善)

\textbf{Proposition 7} 若$S \approxeq^s_{\mathsf{Th}} T$,则对于每一个代换$\rho$,$S\rho \approxeq^s_{\mathsf{Th}} T\rho$

\textit{Proof.} 可以通过VPC的Lemma 3证明。
