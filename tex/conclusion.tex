% !TeX root = ../thesis.tex
\chapter{结论}


\section{总结与评价}

\section{进一步工作}
% 我们在定义$\mathbb{RVPC}_{\mathsf{Th}}$的观察等价性时,
% 将符号互模拟关系的全集定义为观察等价性。
% 但这样定义对于部分$\mathcal{T}_{\mathbb{RVPC}_{\mathsf{Th}}}$是存在问题的:
% 对于$A\in \mathcal{T}_{\mathbb{RVPC}_{\mathsf{Th}}}$,
% 若$A$关于$\varphi\mathcal{E}$的条件等价树的所有分支都是无限的,即$A$没有叶子结点,
% 我们称这样的条件等价树为\textit{发散}。
% 那么我们就不能找到$A$在$\varphi$条件下的条件$l$-转移和条件$q$-转移,
% 进而无法找到它与其他$\mathcal{T}_{\mathbb{RVPC}_{\mathsf{Th}}}$的符号互模拟关系。
% 在Uniform Approach中提出了codivergent的概念,
% 我们的想法也是用\textit{发散}的条件等价森林去模拟\textit{发散}的条件等价树,
% 但这部分工作还没有完成,因此本文有关等价性的理论目前只能应用于非发散的$\mathbb{RVPC}_{\mathsf{Th}}$。

出于展示应用的目的,对Gossip-Style Membership Protocol的建模并不是最优的,
需要改进的地方有: