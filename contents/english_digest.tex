% !TEX root = ../thesis.tex

\begin{digest}
  With the rapid development of parrallel and distributed computer system,
  concurrency theory plays a key role in the theory of programs.
  Randomization is also am important part in Computer Science since it is a computational property. 
  Recently, Fu proposed a uniform approach
  used to turn a process model into a randomized extension.
  Since it is a model independent approach,
  we can use it to derive a probabilistic extension of 
  any other process model.

  A value-passing calculus is a process calculus where the content of communications are values chosen from some data domain,
  and the propositions appearing in the conditionals are formulas constructed from a logic.
  It can be applied in modeling and analyzing communicating processes,
  biological processes and other real-world problems with value-passing characteristics.
  In most studys, the data domain, logic and even the computation of functions in a value-passing calculus are dependent on undefined oracles.
  Those oracles make it hard for those value-passing calculus to measure their expressiveness.
  However,
  Fu proposed a value-passing calculus, called $\mathbb{VPC}_{\mathsf{Th}}$, successfully avoid such oracles, 
  using a first order thoery to decide the bool expression and a turing complete numeric system to derive the outcomes of functions.
  Considering its powerful expressiveness,
  we 


\end{digest}
