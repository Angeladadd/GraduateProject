% !TEX root = ../main.tex

\begin{digest}
  With the rapid development of parallel and distributed computer system, 
  concurrency theory has become an important branch of the theory of programs. 
  As an important extant to concurrency theory, 
  probabilistic process has been widely researched. 
  Recently, Fu proposed a Uniform Approach used to turn a process model into a randomized extension. 
  Since it is a model independent approach, 
  we can use it to derive a probabilistic extension of any other process model. 
  Fu expressed Uniform Approach by demonstrating how to define the grammar and syntax of RCCS, 
  a randomized version of CCS, 
  and provide us a way to define bisimulation and equivalence on RCCS as well.

  A value-passing calculus is a process calculus where the content of communications are values chosen from some data domain, 
  and the propositions appearing in the conditionals are formulas constructed from a logic. 
  It can be applied in modeling and analyzing communicating processes, 
  biological processes and other real-world problems with value-passing characteristics. 
  In most studies of value-passing calculus, there exits some oracles providing data domain, 
  logic decision and even computation of functions in value-passing processes. 
  Those oracles are usually remained undefined making it hard to analyze the expressiveness of those models. 
  However, Fu proposed a value-passing calculus, 
  called VPC_Th, using a first order theory to decide the bool expression and a turing complete numeric system to derive the outcomes of functions, 
  successfully avoid such oracles.
Consider its great expressiveness, 
we decide to use Uniform Approach to extend VPC_Th into a probabilistic version, 
called RVPC_Th. Hopefully, RVPC_Th could help to model and analyze some meaningful real-world process such as communication and network security etc.

\end{digest}
