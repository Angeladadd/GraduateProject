% !TEX root = ../main.tex

\begin{digest}
<<<<<<< HEAD
  With the rapid development of parallel and distributed computer system, concurrency theory has become an important branch of the theory of programs. Research on concurrency theory deepens people’s understanding of concurrent system. Some of the research outcomes have already been used in mainstream programming languages with concurrency features, like Ada and Java. Process calculus is a kind of formal methods using algebraic methods to study concurrent system, such as CCS, CSP and ACP. As a model for describing concurrent systems, process calculus has been extensively studied and successfully applied to the specification, design, analysis, and verification of actual systems.
  
  Modern computer systems, which are open, distributed and interactive, have both nondeterministic behaviors and random choices. In order to use simple, easy-to-use formal methods to describe complex concurrent systems, and to model and analyze concurrent systems, we usually use statistical behavioral characteristics of non-deterministic behavior. Therefore, it is meaningful to introduce the concept of randomness in the concurrent process model. As an important extant to concurrency theory, probabilistic process has been widely researched. Recently, Fu proposes a Uniform Approach used to turn a process model into a randomized extension. Since it is a model independent approach, we can use it to derive a probabilistic extension of any other process model. Fu expresses Uniform Approach by demonstrating how to define the grammar and syntax of RCCS, a randomized version of CCS, and provides us a way to define bisimulation and equivalence on RCCS as well.

  A value-passing calculus is a process calculus where the content of communications are values chosen from some data domain, and the propositions appearing in the conditionals are formulas constructed from a logic. It can be applied to modeling and analyzing communicating processes, biological processes and other real-world problems with value-passing characteristics. In most studies of value-passing calculus, there exits some oracles providing data domain, logic decision and even computation of functions in value-passing processes. Those oracles are usually remained undefined making it hard to analyze the expressiveness of those models. However, Fu proposes a value-passing calculus, called $\mathbb{VPC}_{\mathsf{Th}}$, using a first order theory to decide the bool expression and a turing complete numeric system to derive the outcomes of functions, which successfully avoid such oracles. 
  
  Consider $\mathbb{VPC}_{\mathsf{Th}}$’s great expressiveness, we decide to use Uniform Approach to extend $\mathbb{VPC}_{\mathsf{Th}}$ into a probabilistic version, called $\mathbb{RVPC}_{\mathsf{Th}}$. Hopefully, $\mathbb{RVPC}_{\mathsf{Th}}$ could help to model and analyze some meaningful real-world process such as communication and network security etc. With the aim of benefit mankind, we apply Uniform Approach to $\mathbb{VPC}_{\mathsf{Th}}$ and get the grammar and symbolic transition syntax of $\mathbb{RVPC}_{\mathsf{Th}}$, specifically, we add random choice operator to $\mathbb{RVPC}_{\mathsf{Th}}$ as well as a transition rule to random choice. 
  
  Equivalence of processes is a basic topic of the theory of programs. In the field of process calculus, we usually use bisimulation to describe the equivalence of process models. Bisimulation has been studied for many years since process calculus came into being. Representative work includes Milner’s weak bisimulation and van Glabbeek and Weijland’s branching bisimulation. As for equivalence of probabilistic process, there already exits some research on full probabilistic processes, finite states probabilistic processes etc. However, it is hard to use branching bisimulation or weak bisimulation to describe the equivalence of value-passing process because of the conditional operator. In this way, scholars came up with a symbolic bisimulation for value-passing calculus. Fu also defined the symbolic bisimulation of $\mathbb{VPC}_{\mathsf{Th}}$. 
  
  Uniform Approach proposes a random version of branching bisimulation to describe the equivalence of RCCS. Similarly, we can use the method in Uniform Approach to define a random version of symbolic bisimulation for our $\mathbb{RVPC}_{\mathsf{Th}}$. The core idea to derive a random version of a certain bisimulation is to find out a random version of state-preserving silent transition. Uniform Approach uses Epsilon tree to construct a random branching bisimulation. The main difficulty of using Uniform Approach to define $\mathbb{RVPC}_{\mathsf{Th}}$’s symbolic bisimulation is still the conditional operator. With the aim of eliminate the influence of conditional operator, we propose the conditional equivalence class and conditional Epsilon tree. After defining the conditional l-transition and q-transition, we propose a random symbolic bisimulation. The essence of our symbolic bisimulation is that we can find a division of a certain condition and use conditional epsilon trees corresponding to each element in the division to simulate a conditional epsilon tree. As for some special processes that have a conditional epsilon tree with all branches infinite, we define the codivergence of $\mathbb{RVPC}_{\mathsf{Th}}$ for those processes. Then we define the observance equivalence of $\mathbb{RVPC}_{\mathsf{Th}}$ as the largest codivergent and symbolic bisimulating relation. We manage to prove the congruence of our observance equivalence. Meanwhile, we also provide some interesting examples to annotate the concepts mentioned above.
  
  In addition, for showing our newly proposed random value-passing process model is feasible to model and analyze some real-world processes, we decide to model a real-world process as a demo. Since gossip protocol is interesting, widely known and easy to understand and failure detection is a mainstream topic of cloud computing, we choose to model and analyze a communicating process based on a failure detection mechanism, Gossip-Style Membership Protocol. There’s something with randomization inside gossip protocol that every gossip time, each node will pick k other nodes in a cluster to send gossip message. It is a good property for us to apply our random value-passing process model. We firstly construct a peer-to-peer system where nodes all use gossip protocol to communicate with each other. Since gossip is used to implement multicast in a group, we also define a specification of a multicasting system and prove that the multicasting system is symbolic bisimulated with our peer-to-peer system based on gossip protocol. Next, we adjust the definition of the former peer-to-peer system and using $\mathbb{RVPC}_{\mathsf{Th}}$ to define a membership system interacting with nodes connected to the communicating network of the peer-to-peer system. Finally, we use Golang to implement our peer-to-peer system and use html and javascript to show the result. Result shows that our peer-to-peer system implemented using $\mathbb{RVPC}_{\mathsf{Th}}$ runs correctly, which implies that $\mathbb{RVPC}_{\mathsf{Th}}$ is feasible to be used in modeling and analyzing real-world process with value-passing characteristics, such as communicating processes and biology processes etc. Hopefully, after we define and prove several concepts of $\mathbb{RVPC}_{\mathsf{Th}}$ and demonstrate with an application of it, modeling and analyzing real-world process with value-passing characteristics could be easier and even a routine work.

  Although there still exists something not well considered in our model, our expansion of the value-passing process model, for one hand, further supports the model independence of the Uniform Approach. On the one hand, it proves the feasibility of applying the random value-passing process model to the modeling and analysis of concurrent systems with value-passing characteristics.

\end{digest}
