% !TeX root = ../thesis.tex
\chapter{结论}

本文在传值进程模型$\mathbb{VPC}_{\mathsf{Th}}$的基础上,
使用一种模型无关的概率扩展方法,
扩展了$\mathbb{VPC}_{\mathsf{Th}}$的语法和迁移语义,
增加了随机选择操作子,
使其除非确定性选择之外还可以做随机选择,
我们称这一模型为随机传值进程模型,记为$\mathbb{RVPC}_{\mathsf{Th}}$。

同时,为了给出$\mathbb{RVPC}_{\mathsf{Th}}$的观察等价性的定义,
我们使用Uniform Approach的方法定义了条件等价集、条件等价树(森林)、
条件$l$转移和条件$q$转移,进而定义了$\mathbb{RVPC}_{\mathsf{Th}}$下
的符号互模拟。
条件等价树的本质是概率化了$\mathbb{VPC}_{\mathsf{Th}}$中的
状态保持的静态迁移$A\Rightarrow_{\varphi}A'=A$,
$\mathbb{RVPC}_{\mathsf{Th}}$中的符号互模拟的本质是
用条件等价树中条件$\varphi$对应的划分$\{\varphi_i\}_{i\in I}$所
对应的条件等价森林模拟条件等价树中的条件$l$转移和条件$q$转移。
我们将共发散的符号互模拟关系的全集定义为$\mathbb{RVPC}_{\mathsf{Th}}$的观察等价性,
并给出了观察等价性的同余性证明。

最后,我们以一种云计算协议Gossip-Style Membership Protocol为例,
给出了使用$\mathbb{RVPC}_{\mathsf{Th}}$建模通信过程,乃至其他现实问题的
方法和过程,并使用$\mathbb{RVPC}_{\mathsf{Th}}$的符号互模拟
给出了Gossip协议与我们定义的多播规约的观察等价性。
给出Gossip-Style Membership Protocol的模型的同时,
我们使用Go语言实现了这个模型,并分析了Go语言特性对应
部分$\mathbb{RVPC}_{\mathsf{Th}}$操作子的实现。
事实证明$\mathbb{RVPC}_{\mathsf{Th}}$在建模和分析通信过程时具有一定的可行性。
我们希望$\mathbb{RVPC}_{\mathsf{Th}}$可以为
含有传值、计算性质的现实问题的建模和分析提供有效、可行的方法。

\section{待改进的工作}
随机传值进程模型中的随机性可以体现在两个方面:\textit{内容随机性}和\textit{通道随机性},
内容随机性体现在值的随机性,外界向进程传的值可能会在某一个值域中符合某种概率分布,或近似符合某种概率分布;
而通道随机性体现在进程对通信通道的选择是随机的,如在第~\ref{ch:gossip}章中,Gossip协议会周期性的随机选择通信的对象。
由于Uniform Approach中只关注了通道的随机性,
在本文对随机传值进程模型的定义中,我们也只关注了通道的随机性。
这样做有一定的合理性:对于一个与具体问题无关的形式化方法,
我们无法给定值的一个具体的随机分布。
对于具体的问题,我们可以在$\mathbb{RVPC}_{\mathsf{Th}}$的基础上考虑值的随机分布。

但遗憾的是,
在第~\ref{ch:gossip}章中我们给出的应用,
包括在第~\ref{ch:rvpc}章中的简单的例子,
实际上都是$\mathbb{RVPC}_{\mathsf{Th}}$
上的\textit{进程},
由于系统没有与外界的值的通信,整个系统中没有自由变元,
因此依旧是不需要考虑内容随机性的模型。
这是本文在构思上一个欠考虑的地方,
实际上建模一个与外界有值传递的系统对于本文的意义会更大一些。

进一步的研究可以关注内容随机性。
对于内容的随机性的实现,MILNER R在Communication and Concurrency\cite{Milner_CCS}
中在将书中的提供了一个思路:
为了建模传值的CCS,MILNER R将Value-passing CCS转化为CCS时
将$S=a(x).T$转为$S=\sum_{v\in V} a_v.T\{v/x\}$,其中$V$是给定值的集合。
我们也可以用$S=\bigoplus_{v\in V}p_v.\tau.a(v).T\{v/x\}$的方法来建模值的随机性。