% !TEX root = ../main.tex

\begin{abstract}
  随着并行和分布式计算机系统的迅速发展,
  并发程序理论成为程序理论的一个重要分支。
  作为经典并发理论的扩展,概率进程被广泛研究。
  最近,一个对并发进程模型进行概率化扩展的通用方法被提出,
  我们使用这个通用方法对经典进程模型——并发传值进程模型进行了概率扩展。
  我们提出了概率扩展后得到的随机传值进程模型的语法和转移语义,
  并给出了随机传值进程模型互模拟关系、等价性的定义以及同余性的证明。
  随机传值进程模型理论上可以用于具有传值特点的现实问题的建模和分析。
  我们使用随机传值进程模型建模并模拟实现了基于云计算协议Gossip-Style Membership协议的通信过程,
  证明了随机传值进程模型对于并发传值通信过程的建模和分析具有一定的可行性。
\end{abstract}

\begin{abstract*}
  With the rapid development of parallel and distributed computer systems,
  concurrency theory has become an important part of the theory of programs.
  As an extension of the classic concurrency theory,
  probabilistic processes has been widely studied for many years.
  Recently, a uniform approach used to derive the probabilistic extension of 
  concurrency process models has been proposed.
  We can get a random version of value-passing calculus,
  one of the classic process calculus, called random value-passing calculus, 
  by using the uniform approach.
  In this paper, we propose the grammar and transition semantics of random value-passing calculus
  as well as its bisimulation and equivalence relation.
  We also make the new equivalence relation to be congruent. 
  Random value-passing calculus can theoretically 
  be applied in modeling and analyzing real-world problems with value-passing characteristics.
  We use random value-passing calculus to model and implement
  a communication system based on the Gossip-Style Membership Protocol, a cloud computing protocol,
  which proves that this model has certain feasibility 
  for the modeling and analysis of concurrent value-passing communication processes.
\end{abstract*}
