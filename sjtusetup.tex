% !TEX root = ./main.tex

\sjtusetup{
  %
  %******************************
  % 注意:
  %   1. 配置里面不要出现空行
  %   2. 不需要的配置信息可以删除
  %******************************
  %
  % 信息录入
  %
  info = {%
    %
    % 标题
    %   可使用“\\”命令手动控制换行
    %
    title           = {ON UNIFORM APPROACH TO RANDOM PROCESS MODEL},
    title*          = {On Uniform Approach to Random Process Model},
    %
    % 关键词
    %
    keywords        = {并发理论,随机进程,并发传值进程,互模拟关系,观察等价性},
    keywords*       = {concurrency theory, probabilistic process, value-passing calculus, bisimulation, observation equivalence},
    %
    % 姓名
    %
    author          = {孙晨鸽},
    author*         = {Sun Chenge},
    %
    % 指导教师
    %
    supervisor      = {傅育熙教授},
    supervisor*     = {Prof. Fu Yuxi},
    %
    % 副指导教师
    %
    assisupervisor  = {龙环教授},
    assisupervisor* = {Prof. Long Huan},
    %
    % 学号
    %
    id              = {516030910421},
    %
    % 学位
    %   本科生不需要填写
    %
    degree          = {工学学士},
    degree*         = {Bachelor of Engineering},
    %
    % 专业
    %
    major           = {计算机科学与技术},
    major*          = {Computer Science},
    %
    % 所属院系
    %
    department      = {计算机科学与工程系},
    department*     = {Depart of Computer Engineering},
    %
    % 课程名称
    %   仅课程论文适用
    %
    % coursename      = {某某课程},
    %
    % 答辩日期
    %   使用 ISO 格式;默认为当前时间
    %
    % date            = {2014-12-17},
    %
    % 资助基金
    %
    % fund  = {
    %           {国家 973 项目 (No. 2025CB000000)},
    %           {国家自然科学基金 (No. 81120250000)},
    %         },
    % fund* = {
    %           {National Basic Research Program of China (Grant No. 2025CB000000)},
    %           {National Natural Science Foundation of China (Grant No. 81120250000)},
    %         },
  },
  %
  % 风格设置
  %
  style = {%
    %
    % 本科论文页眉 logo 颜色
    %   默认为黑色
    %
    header-logo-color = red,
  },
  %
  % 名称设置
  %
  name = {%
    % publications      = {攻读学位期间完成的论文},
  },
}

% 参考文献支持宏包
\usepackage[backend=biber,style=gb7714-2015,gbpub=false,gbpunctin=false]{biblatex}
% 导入参考文献数据库
\addbibresource{bibdata/thesis.bib}

% 定义图片文件目录与扩展名
\graphicspath{{figures/}}
\DeclareGraphicsExtensions{.pdf,.eps,.png,.jpg,.jpeg}

% 确定浮动对象的位置,可以使用 [H],强制将浮动对象放到这里(可能效果很差)
% \usepackage{float}

% 固定宽度的表格
% \usepackage{tabularx}

% 表格中支持跨行
\usepackage{multirow}

% 表格中数字按小数点对齐
\usepackage{dcolumn}
\newcolumntype{d}[1]{D{.}{.}{#1}}

% 附带脚注的表格
\usepackage{threeparttable}

% 算法环境宏包
\usepackage[ruled,vlined,linesnumbered]{algorithm2e}
% \usepackage{algorithm}

% 代码环境宏包
\usepackage{listings}
\lstnewenvironment{codeblock}[1][]
  {\lstset{style=lstStyleCode,#1}}{}

% 国际单位制宏包
\usepackage{siunitx}

% 定理环境宏包
\usepackage{ntheorem}
% \usepackage{amsthm}

% 绘图宏包
\usepackage{tikz}

% 一些文档中用到的 logo
\usepackage{hologo}
\newcommand{\XeTeX}{\hologo{XeTeX}}
\newcommand{\BibLaTeX}{\textsc{Bib}\LaTeX}

% 借用 ltxdoc 里面的几个命令。
\def\cmd#1{\cs{\expandafter\cmd@to@cs\string#1}}
\def\cmd@to@cs#1#2{\char\number`#2\relax}
\DeclareRobustCommand\cs[1]{\texttt{\char`\\#1}}

\newcommand*{\meta}[1]{{%
  \ensuremath{\langle}\rmfamily\itshape#1\/\ensuremath{\rangle}}}
\providecommand\marg[1]{%
  {\ttfamily\char`\{}\meta{#1}{\ttfamily\char`\}}}
\providecommand\oarg[1]{%
  {\ttfamily[}\meta{#1}{\ttfamily]}}
\providecommand\parg[1]{%
  {\ttfamily(}\meta{#1}{\ttfamily)}}
\providecommand\pkg[1]{{\sffamily#1}}

% 自定义命令

% E-mail
\newcommand{\email}[1]{\href{mailto:#1}{\texttt{#1}}}

% hyperref 宏包在最后调用
\usepackage{hyperref}
